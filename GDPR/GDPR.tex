\documentclass[hidelinks,12pt,a4paper]{article}
\usepackage[italian]{babel}
\usepackage[utf8]{inputenc}
\usepackage{fourier} 

% To avoid GitHub Action error
\usepackage{hyperref}

% Images
\usepackage{graphicx}
\usepackage{caption}
\usepackage{subcaption}
\usepackage{float}
\graphicspath{ {../Images} }

% Minipages in the same line
\usepackage{tabularx}

% Create page border 
\usepackage{tikz}
\usepackage[framemethod=TikZ]{mdframed}
\newmdenv[tikzsetting={draw=black, line width=6pt}, backgroundcolor=none]{roundCornerPage}

% Adjust margins
\usepackage{changepage}
\usepackage{geometry}

% Dummy text
\usepackage{lipsum} 

% To have conditional text
\usepackage{ifthen}

% License
\usepackage[
type={CC},
modifier={by-nc-sa},
version={4.0},
]{doclicense}

%---------------------------------- Setting params -----------------------------------%

%%%%% Set this value true to have two informations part on top of document %%%%% 
\newbool{splitedInfo}
\setbool{splitedInfo}{true}
%%%%% 

%%%%% Set this value true to have circular logo image %%%%% 
\newbool{circularLogo}
\setbool{circularLogo}{true}
%%%%% 

%%%%% Set logo image %%%%% 
\newcommand{\logoImage}{\includegraphics[scale=0.025]{Lorem_Ipsum_Logo.jpg}}
% Setting circle size
\newcommand{\circleSize}{1.5cm}
% Center image manually 
\newcommand{\rightMargin}{0mm}
\newcommand{\leftMargin}{1.5cm}
\newcommand{\vertical}{-1cm}
%%%%% 

%%%%% Set informations %%%%% 
\newcommand{\informations}{\lipsum[1][1-4]}
%%%%% 

%%%%% Set second informations column %%%%% 
\newcommand{\informationsTwo}{\lipsum[1][1-4]}
%%%%% 

%---------------------------------- End setting params -----------------------------------%

\begin{document}
	% Remove page number
	\pagestyle{empty}
	
	\title{\textbf{GDPR template}}
	\author{Francesco Rombaldoni}
	\date{}
	
	\maketitle
	\newpage
	
	%---------------------------------------------------------------------%
	\newgeometry{left=0.1cm, right=0.1cm, top=0.1cm, bottom=0.1cm}
		\begin{roundCornerPage}[roundcorner=15pt]
			\begin{minipage}[t][0.97\paperheight][t]{0.9\paperwidth}%0.9\paperwidth%0.9\paperheight
				
				\begin{minipage}[t][0.18\textwidth][t]{\textwidth}
					\ifthenelse{\boolean{splitedInfo}}
					{
						\begin{tabularx}{\textwidth}{XXX}
							{
							% First of three columns
							\informations
							}&{
							% Second of three columns
							\ifthenelse{\boolean{circularLogo}}
							{
									\begin{center}
									\begin{adjustwidth}{\leftMargin}{\rightMargin}
										\vspace*{\vertical}
										\begin{tikzpicture}
											\clip (0,0) circle (\circleSize) node {\logoImage};
										\end{tikzpicture}
									\end{adjustwidth}
								\end{center}
							}{
								\begin{center}
										\begin{adjustwidth}{\leftMargin}{\rightMargin}
											\vspace*{\vertical}
											\logoImage
										\end{adjustwidth}
								\end{center}
							}
							}&{
							% Last Column
							\hspace*{1cm}
							\begin{minipage}[t][0.18\textwidth][t]{\linewidth}
								\informationsTwo
							\end{minipage}
							}
						\end{tabularx}
					}{
						\begin{tabularx}{\textwidth}{XX}
							{
								% First column
								\ifthenelse{\boolean{circularLogo}}
								{
									\begin{center}
										\begin{adjustwidth}{\leftMargin}{\rightMargin}
											\vspace*{\vertical}
											\begin{tikzpicture}
												\clip (0,0) circle (\circleSize) node {\logoImage};
											\end{tikzpicture}
										\end{adjustwidth}
									\end{center}
								}{
									\begin{center}
										\begin{adjustwidth}{\leftMargin}{\rightMargin}
											\vspace*{\vertical}
											\logoImage
										\end{adjustwidth}
									\end{center}
								}
							}&{
								% Last column
								\informations
							}
						\end{tabularx}
					}
				\end{minipage}
				% End line
				\vspace*{5mm}
				\hspace*{-1.5mm}
				\noindent\tikz\draw[line width=4pt, line cap=round, black!80](0,0) -- node[] {} (1.065\linewidth,0);
				%===== Bottom part ========%
				
				\begin{center}
					\Large{\textbf{Informativa sulla privacy}}
				\end{center}
				
				Gentile cliente, ai sensi dell'articolo 13 del Regolamento UE 2016/679 (anche detto "GDPR 2016/679") recante le disposizioni a tutela dei dati personali delle persone e altri soggetti ed in relazione ai dati di cui entreremo in possesso, desideriamo informarla di quanto segue: 
				
				\begin{tabularx}{\linewidth}{XX}
					{ %First column
						\begin{enumerate}
							\item[\large{\textbf{1.}}] \large{\textbf{I dati raccolti e la loro finalità}}\medskip \newline
							\footnotesize
							Dati raccolti per poter accedere al servizio con il fine di: "Inserire fine per la raccolta"
							\begin{itemize}
								\item Nome.
								\item Cognome.
								\item Sesso.
								\item Data di nascita.
								\item Indirizzo di casa.
								\item Indirizzo e-mail.
								\item Numero di telefono o di cellulare.
							\end{itemize}
							\item[\large{\textbf{2.}}]  \large{\textbf{Trattamento dei dati raccolti}} \newline
							\footnotesize
							\vspace*{-5mm}
							\begin{itemize}
								\item Tutti i dati sono trattati in conformità alle vigenti normative in materia di privacy (Reg. UE 2016/679).
								\item Tutti i dati sono trattati in modo lecito, corretto e trasparente nei confronti dell'interessato, nel rispetto dei principi generali previsti dall'Art.5 del GDPR.
								\item Specifiche misure di sicurezza sono osservate per prevenire la perdita dei dati, usi illeciti, o non corretti ed accessi	non autorizzati (GDPR, Art.32).
								\item Nessun dato è condiviso esternamente con un ente che non sia la "Nome ente" e con del personale che non sia interno del suddetto ente.
							\end{itemize}
						\end{enumerate}
					}&{
						\begin{enumerate}
							\item[\large{\textbf{3.}}]  \large{\textbf{Diritti degli interessati}} \newline
							\footnotesize
							\vspace*{-5mm}
							\begin{itemize}
								\item Diritto di richiedere la presenza e l’accesso a dati personali che lo riguardano (Art.15 "Diritto di accesso").
								\item Diritto di ottenere la rettifica/integrazione di dati inesatti o incompleti (Art.16 "Diritto di rettifica").
								\item Diritto della cancellazione dei dati (Art.17 "Diritto alla cancellazione"). In caso di cancellazione dei dati obbligatori si perde l'iscrizione al servizio.
								\item Diritto di ottenere la limitazione del trattamento (Art.18 "Diritto alla limitazione").
								\item Diritto di ricevere in formato strutturato i dati che lo riguardano (Art.20 "Diritto alla portabilità").
								\item Diritto di opporsi al trattamento (Art.21 "Diritto di opposizione").
								\item Diritto di revocare un consenso precedentemente prestato.
								\item Diritto di presentare, in caso di mancato riscontro, un reclamo all'Autorità Garante per la protezione dei dati.
							\end{itemize}
							\item[\large{\textbf{4.}}]  \large{\textbf{Periodo di conservazione dei dati}} \newline
							\footnotesize
							\vspace*{-5mm}
							\begin{itemize}
								\item Tutti i dati raccolti sono conservati fino a quando l'interessato è cliente di "Nome ente". In caso di espressa volontà da parte dell'interessato di cancellare la propria iscrizione, varranno conseguentemente cancellati i dati dell'interessato. Stessa cosa si verifica dopo un anno di inattività dello stesso.
							\end{itemize}
						\end{enumerate}
						
						\vspace*{\fill}
						\begin{enumerate}
							\item[]  \small{\textbf{Firmando l'interessato esprime il consenso al trattamento dei suoi dati personali nei termini definiti nell'informativa che precede e dichiara di aver letto e compreso l'informativa.}} \newline
							\begin{center}
								\scriptsize
								Firma leggibile dell'interessato
								\vspace*{5mm}
								\hspace*{-20mm}
								\newline
								\rule{50mm}{0.30mm}
								
							\end{center}
						\end{enumerate}
						
					}
				\end{tabularx}
				
				\end{minipage}
			\end{roundCornerPage}
		\restoregeometry
	%---------------------------------------------------------------------%
	
	\newpage
	
	%Print license
	\vspace*{\fill}
	\doclicenseThis
	
\end{document}
